\documentclass[13pt]{beamer}

\usepackage{pgfpages}
\usepackage{minted}
\usepackage[T1]{fontenc}
\usepackage[default]{lato}
\usepackage{lmodern}
\usepackage[english]{babel}
\usepackage[utf8x]{inputenc}
\usepackage{graphicx}
\usepackage{dirtytalk}
\usepackage{lipsum}

\usetheme{default} % or try Darmstadt, Madrid, Warsaw, ...
\usecolortheme{default} % or try albatross, beaver, crane, ...
\usefonttheme{default} % or try serif, structurebold, ...
\setbeamertemplate{navigation symbols}{}
\setbeamertemplate{caption}[numbered]

% For presenter notes
%\setbeameroption{show notes on second screen}


\addtobeamertemplate{navigation symbols}{}{%
  \usebeamerfont{footline}%
  \usebeamercolor[fg]{footline}%
  \hspace{1em}%
  \insertframenumber/\inserttotalframenumber
  }

\newcommand\Wider[2][3em]{%
\makebox[\linewidth][c]{%
  \begin{minipage}{\dimexpr\textwidth+#1\relax}
  \raggedright#2
  \end{minipage}%
  }%
}
  
\title{Type Classes in Scala}
\subtitle{What? Why? How?}
\author{Daniel Reigada}
\institute{IST - Programming Languages}
% \date{Date of Presentation}

% pygmentize -L styles
\usemintedstyle{lovelace}

\usepackage{xcolor}


\color{white}

\definecolor{backgroundColor}{HTML}{FFFFFF} 
\definecolor{titleBackgroundColor}{HTML}{15414C} 
\definecolor{titleColor}{HTML}{FFFFFF} 
\definecolor{textColor}{HTML}{4A5659} 


\definecolor{codeBackground}{HTML}{e7e7d6} 
\definecolor{codeBorderColor}{HTML}{e7e7d6} 

\setbeamercolor{background canvas}{bg=backgroundColor}

\setbeamercolor{titlelike}{parent=structure, bg=titleBackgroundColor, fg=titleColor}
\setbeamercolor{normal text}{fg=textColor}


\makeatletter
\renewenvironment{minted@colorbg}[1]{
\setlength{\fboxsep}{\z@}
\def\minted@bgcol{#1}
\noindent
\begin{lrbox}{\minted@bgbox}
\begin{minipage}{\linewidth}}
{\end{minipage}
\end{lrbox}%
\colorbox{\minted@bgcol}{\usebox{\minted@bgbox}}}
\makeatother

\newminted[scalaCode]{scala}{fontsize=\footnotesize}
\newminted[haskellCode]{haskell}{fontsize=\scriptsize, xleftmargin=-20pt}%, bgcolor=codeBackground, frame=single, framesep=30\fboxrule, rulecolor=\color{textColor}}
\newminted[javaCode]{java}{fontsize=\scriptsize}%, bgcolor=codeBackground, frame=single, framesep=30\fboxrule, rulecolor=\color{textColor}}


\begin{document}

\maketitle

\section{Introduction}

\begin{frame}{Questions I would like answer}

  \begin{enumerate}
    \item What is type class?
    \item Why should you use type classes?
    \item How to use type class in Scala?
  \end{enumerate}

\end{frame}

\begin{frame}{What is a type class?}
  \note{
    Well, but object oriented classes are just that!

    Yes, they a mean to the same end (or at least similar ends)
    What's so different about type classes?
  }

  From Wikipedia: \\
  \say{A type system construct that supports ad hoc polymorphism.} \\ \\

  \onslide<2, 3>{But classes in object oriented languages are just that!} \\ \\

  \onslide<3>{What's so different about type classes?}
  
\end{frame}

\begin{frame}[fragile]{What is a type class?}
  \note{
    Type classes are similiar to interfaces in the way they add
    information and operations on top of types.

    But they differ in the way they are defined and implemented.\\
    With interfaces the class needs to extend the interface and implement the methods. \\
    While with typeclasses the type and the typeclass for the type are deffined separatly. \\

    Implementations of typeclasses are called Instances.
  }
   
  \begin{columns}[c]
    \hspace{-10pt}
    \column{(\dimexpr\paperwidth-80pt)/2}
      \begin{haskellCode}
        class Writable a where
          write :: a -> String




        data Name = Name String String

        instance Writable Name where
          write (Name firstName lastName) =
            firstName ++ " " ++ lastName
      \end{haskellCode}

    \column{(\dimexpr\paperwidth+30pt)/2}
      \begin{javaCode}
        interface Writable {
          public String write();
        }


        
        class Name implements Writable {
          String firstName, lastName;

          public String write(){
            return firstName + " " + lastName;
          }
        }
      \end{javaCode}
  \end{columns}



  % Type classes are similar to interfaces.

  % Type classes instances definitions are separed from the type defitiions.



\end{frame}

\begin{frame}[fragile]{Why type classes? - Modularity}
  \note{
  }

  Why should a list be concerned in implementing RandomAccess, Cloneable and Serializable? \\

  \begin{javaCode}
    public class ArrayList<E> extends AbstractList<E>
        implements List<E>, RandomAccess, Cloneable, java.io.Serializable {
        /* ... */
      }
  \end{javaCode}
\end{frame}

\begin{frame}[fragile]{Why type classes? - Extending existing types}
  \note{
    We can add functions over existing types without changing the type itself
  }

  \begin{haskellCode}
    data MyType = MyType String Int

    -- somewhere else, far from the definition:
    instance Show MyType where
      show (MyType s i) = s ++ ", " ++ (show i)
  \end{haskellCode}


\end{frame}

\begin{frame}[fragile]{Why type classes? - Generic code}
  \note{
    That means it's really hard to make things like:
       add :: t -> t -> t with an interface
  }

  We would like add to only add numbers of the same type: \\

  \begin{javaCode}
    interface Num {
      Num add (Num i);
    }

    class Int implements Num {
      public Int add(Num i) { /*...*/ }
      
      // does not override add:
      // public Int add(Int i) { /*...*/ }
    }
  \end{javaCode}

\end{frame}

\begin{frame}[fragile]{Why type classes? - Generic code}
  \note{
    Boilerplate and harder to understand code
  }

  Solution in java: \\

  \begin{javaCode}
    interface Num<Impl> {
      Impl add (Impl i);
    }
    
    class Int implements Num<Int> {
      public Int add(Int i) { /*...*/ }
    }
  \end{javaCode}

\end{frame}

\begin{frame}[fragile]{Why type classes? - Generic code}
  \note{
    Boilerplate and harder to understand code
  }

  Solution in Haskell: \\

  \begin{haskellCode}
      class Numero t where
        add :: t -> t -> t

      instance Num Int where
        add a b = a + b
  \end{haskellCode}

\end{frame}


\begin{frame}[fragile]{How to use type classes? (in Scala)}
  \note {
  }

  A simple type class:
  \begin{scalaCode}
    trait Show[A] {
      def show(a: A): String
    }

    val intInstance = new Show[Int] {
      override def show(a: Int) = s"Int: " + a
    }

    intInstance.show(123)
    // res0: String = Int: 123
    
  \end{scalaCode}
\end{frame}

\begin{frame}[fragile]{Aside - Scala implicits}
  \note {
  }

  \begin{scalaCode}
    def method(str: String)(implicit strImplicit: String) =
      str + strImplicit
    
    method("parameter1")
    // error: could not find implicit value for parameter
    // strImplicit: String
    // method("parameter1")
    // ^
    
  \end{scalaCode}
\end{frame}

\begin{frame}[fragile]{Aside - Scala implicits}
  \note {
  }

  \begin{scalaCode}
    def method(str: String)(implicit strImplicit: String) =
      str + strImplicit
    
    implicit val string = "implicitString"
    method("parameter1")
    // res0: String = parameter1implicitString
    
  \end{scalaCode}
\end{frame}

\begin{frame}[fragile]{How to use type classes? (in Scala)}
  \note {
  }

  \begin{scalaCode}
    trait Show[A] {
      def show(a: A): String
    }

    object Show {
      def show[A](a: A)(implicit sh: Show[A]) = sh.show(a)
    }

    implicit val intInstance = new Show[Int] {
      override def show(a: Int) = s"Int: " + a
    }

    Show.show(123)
    // res0: String = Int: 123
  \end{scalaCode}
\end{frame}


\begin{frame}[fragile]{How to use type classes? (in Scala)}
  \note {
  }

  Using implicit classes:

  \begin{scalaCode}
    implicit class ShowOps[A](a: A)(implicit sh: Show[A]) {
      def show: String = sh.show(a)
    }

    123.show
    // res0: String = Int: 123
  \end{scalaCode}
\end{frame}

\begin{frame}{How to use type classes? (in Scala)}
  \note {
  }
  Live demo/examples
\end{frame}

\end{document}
