\documentclass[13pt]{beamer}

\usepackage{pgfpages}
\usepackage{minted}
\usepackage[T1]{fontenc}
\usepackage[default]{lato}
\usepackage{lmodern}
\usepackage[english]{babel}
\usepackage[utf8x]{inputenc}
\usepackage{graphicx}
\usepackage{dirtytalk}
\usepackage{lipsum}
\usepackage{subfig}

\usetheme{default} % or try Darmstadt, Madrid, Warsaw, ...
\usecolortheme{default} % or try albatross, beaver, crane, ...
\usefonttheme{default} % or try serif, structurebold, ...
\setbeamertemplate{navigation symbols}{}
\setbeamertemplate{caption}[numbered]

% For presenter notes
%\setbeameroption{show notes on second screen}


\addtobeamertemplate{navigation symbols}{}{%
  \usebeamerfont{footline}%
  \usebeamercolor[fg]{footline}%
  \hspace{1em}%
  \insertframenumber/\inserttotalframenumber
  }

\newcommand\Wider[2][3em]{%
\makebox[\linewidth][c]{%
  \begin{minipage}{\dimexpr\textwidth+#1\relax}
  \raggedright#2
  \end{minipage}%
  }%
}
  
\title{Monad Error}
\subtitle{}
\author{Daniel Reigada}
\institute{Codacy}
% \date{Date of Presentation}

% pygmentize -L styles
\usemintedstyle{lovelace}

\usepackage{xcolor}


\color{white}

\definecolor{backgroundColor}{HTML}{FFFFFF} 
\definecolor{titleBackgroundColor}{HTML}{15414C} 
\definecolor{titleColor}{HTML}{FFFFFF} 
\definecolor{textColor}{HTML}{4A5659} 


\definecolor{codeBackground}{HTML}{e7e7d6} 
\definecolor{codeBorderColor}{HTML}{e7e7d6} 

\setbeamercolor{background canvas}{bg=backgroundColor}

\setbeamercolor{titlelike}{parent=structure, bg=titleBackgroundColor, fg=titleColor}
\setbeamercolor{normal text}{fg=textColor}


\makeatletter
\renewenvironment{minted@colorbg}[1]{
\setlength{\fboxsep}{\z@}
\def\minted@bgcol{#1}
\noindent
\begin{lrbox}{\minted@bgbox}
\begin{minipage}{\linewidth}}
{\end{minipage}
\end{lrbox}%
\colorbox{\minted@bgcol}{\usebox{\minted@bgbox}}}
\makeatother

\newminted[scalaCode]{scala}{fontsize=\footnotesize}
\newminted[haskellCode]{haskell}{fontsize=\scriptsize, xleftmargin=-20pt}%, bgcolor=codeBackground, frame=single, framesep=30\fboxrule, rulecolor=\color{textColor}}
\newminted[javaCode]{java}{fontsize=\scriptsize}%, bgcolor=codeBackground, frame=single, framesep=30\fboxrule, rulecolor=\color{textColor}}


\begin{document}

\maketitle

\section{Introduction}

\begin{frame}{Monad error}

  A typeclass defined that allows one to abstract over error-handling monads. \\

  \onslide<2>{But what does that mean?}

\end{frame}

\begin{frame}{Monad error}

  It allows the creation of generic code, that would otherwise return errors wrapped in a specific monad. \\

  \onslide<2>{OKAY! But what does that mean?!?}
\end{frame}

\begin{frame}[fragile]{Monad error - Abstracting the monad}
  
  \onslide<2>{What if \it{denom} is 0 ?} \\ 

  \begin{scalaCode}
    def divide(num: Int, denom: Int): Int = num / denom
  \end{scalaCode}

\end{frame}

\begin{frame}[fragile]{Monad error - Abstracting the monad}

  We can improve this using Option...

  \begin{scalaCode}
    def divide(num: Int, denom: Int): Option[Int] =
      if(denom == 0) None else Some(num / denom)
  \end{scalaCode}

\end{frame}

\begin{frame}[fragile]{Monad error - Abstracting the monad}

  ...or Try...

  \begin{scalaCode}
    def divideTry(num: Int, denom: Int): Try[Int] =
      if (denom == 0) Failure(new Throwable("Division by 0"))
      else Success(num / denom)
  \end{scalaCode}

\end{frame}

\begin{frame}[fragile]{Monad error - Abstracting the monad}

  ...or Future...

  \begin{scalaCode}
    def divideFuture(num: Int, denom: Int): Future[Int] =
      if (denom == 0) Future.failed(new Throwable("Division by 0"))
      else Future.successful(num / denom)
  \end{scalaCode}

\end{frame}

\begin{frame}[fragile]{Monad error - Abstracting the monad}

  ...or Either...

  \begin{scalaCode}
    def divideEither(num: Int, denom: Int): Either[String, Int] =
      if (denom == 0) Left("Division by 0")
      else Right(num / denom)
  \end{scalaCode}

\end{frame}

\begin{frame}[fragile]{Monad error - Abstracting the monad}

  ...or a custom result type (i.e. foundation Response)...

  \begin{scalaCode}
    def divideEither(num: Int, denom: Int): Result[Int] =
      if (denom == 0) Result.error("Division by 0")
      else Result.success(num / denom)
  \end{scalaCode}

\end{frame}


\begin{frame}[fragile]{Monad error - Abstracting the monad}

  ... you get the idea.\\

  What if you are trying to write generic code (i.e. library)?\\
  We should be able to abstract it as much as possible.

  \begin{scalaCode}
    def divide???(num: Int, denom: Int): F[Int] =
      if (denom == 0) ERROR_CASE
      else SUCCESS_CASE
  \end{scalaCode}

\end{frame}


\begin{frame}{Monad error - Abstracting the monad}

  
  \begin{center}
    By the power of the Monad!!! \\
    \vline

    \includegraphics[scale=0.33]{figures/he-man.jpg}
  \end{center}

\end{frame}

\begin{frame}[fragile]{Monad error - Abstracting the monad}

  \begin{scalaCode}
    def divideF[F[_]](num: Int, denom: Int)(
        implicit M: MonadError[F, Throwable]): F[Int] = {
      if (denom == 0) M.raiseError(new Throwable("Division by 0"))
      else M.pure(num / denom)
    }
  \end{scalaCode}

\end{frame}

\begin{frame}[fragile]{Monad error - Abstracting the monad}

  \begin{scalaCode}
    def getNum: Try[Int] = ???
    def getDenom: Try[Int] = ???

    for {
      num <- getNum
      denom <- getDenom
      result <- divideF[Try](num, denom)
    } yield result
  \end{scalaCode}

\end{frame}

\begin{frame}{Monad error - Abstracting the monad}

  \begin{center}
    \includegraphics[scale=0.66]{figures/magic.jpeg}
  \end{center}

\end{frame}


\begin{frame}[fragile]{Monad error - Useful methods}
  \begin{scalaCode}

  // pure
  MonadError[Try, Throwable].pure(1)
  // scala.util.Try[Int] = Success(1)

  // raiseError
  MonadError[Try, Throwable].raiseError(new Throwable("error"))
  // scala.util.Try[Nothing] = Failure(java.lang.Throwable: error)
 
  \end{scalaCode}
\end{frame}


\begin{frame}[fragile]{Monad error - Useful methods}
  \begin{scalaCode}
  // fromEither / fromTry / fromOption / fromValidated

  MonadError[Try, Throwable].fromEither(Right(123)) 
  // scala.util.Try[Int] = Success(123)

  MonadError[Try, Throwable].fromOption(None, new Throwable("empty"))
  // scala.util.Try[Nothing] = Failure(java.lang.Throwable: empty)
 
  \end{scalaCode}

\end{frame}

\begin{frame}[fragile]{Monad error - Useful methods}
  \begin{scalaCode}

  // catchNonFatal
  MonadError[Try, Throwable].catchNonFatal(1 / 0)
  // scala.util.Try[Int] =
  //    Failure(java.lang.ArithmeticException: / by zero)
  \end{scalaCode}

\end{frame}

\begin{frame}[fragile]{Monad error - Useful methods}
  \begin{scalaCode}

  // handleError / handleErrorWith / recover / recoverWith
  MonadError[Try, Throwable]
    .catchNonFatal(1 / 0)
    .recover {
      case _ : ArithmeticException => 0
    }
  // scala.util.Try[Int] = Success(0)

  \end{scalaCode}

\end{frame}


\begin{frame}[fragile]{Monad error - Useful methods}
  \begin{scalaCode}

  MonadError[Try, Throwable]
    .pure(123)
    .attempt
  // scala.util.Try[Either[Throwable,Int]] =
  //    Success(Right(123))

  MonadError[Try, Throwable]
    .raiseError(new Throwable("error"))
    .attempt 
  // scala.util.Try[Either[Throwable,Nothing]] =
  //    Success(Left(java.lang.Throwable: error))

  \end{scalaCode}

\end{frame}

\begin{frame}[fragile]{Monad error - Abstracting the error}
  Our code is still not totally generic. \\
  We are restricted by the error type.

\end{frame}

\begin{frame}[fragile]{Monad error - Abstracting the error}
  \begin{scalaCode}
  trait UIError[A] {
    def errorFromString(str: String): A

    def errorFromThrowable(thr: Throwable): A
  }

  object UIError {
    implicit val throwableInstance: UIError[Throwable] = ??
    implicit val stringInstance: UIError[String] = ???
  }
  \end{scalaCode}


\end{frame}


\begin{frame}[fragile]{Monad error - Abstracting the error}

  \begin{scalaCode}
  def divideF[F[_], E](num: Int, denom: Int)(
    implicit M: MonadError[F, E], Err: UIError[E]): F[Int] = {
    if (denom == 0) M.raiseError(Err.errorFromString("Division by 0")))
    else M.pure(num / denom)
  }

  divideF[Try, Throwable](1, 0)
  // scala.util.Try[Int] = Failure(java.lang.Throwable: Division by 0)
  
  divideF[Either[String, ?], String](1, 0) 
  // scala.util.Either[String,Int] = Left(Division by 0)

  \end{scalaCode}
\end{frame}


\begin{frame}{Questionary time}

  Which one do you feel like?

  \begin{figure}%
    \centering
    \subfloat[]{{\includegraphics[height=4cm]{figures/monad-wat.jpg} }}%
    \qquad
    \subfloat[]{{ \includegraphics[height=4cm]{figures/monad-l-jackson.jpg} }}%
  \end{figure}


\end{frame}

\begin{frame}{Further reading}
  \begin{itemize}
    \item Documentation - https://typelevel.org/cats/api/cats/MonadError.html
    \item Haskell docs (if you are brave enough) - http://hackage.haskell.org/package/mtl-2.2.2/docs/Control-Monad-Error.html
    \item Rethinking MonadError - https://typelevel.org/blog/2018/04/13/rethinking-monaderror.html
  \end{itemize}
\end{frame}

\begin{frame}[plain, c]
  \begin{center}
  {\Huge Q \& A}\\

  {\tiny or invoice-manager demo if we have time}
  \end{center}
\end{frame}

\end{document}
